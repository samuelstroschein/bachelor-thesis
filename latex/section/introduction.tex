\section{Introduction}

Within the last 15 years, multiple key patents regarding 3D printing, also called additive manufacturing (AM) expired \cite{beaman2014additive, bechtold20163d, nuchitprasitchai2017factors}. Simultaneously, research initiatives started developing low-cost 3D printers \cite{beaman2014additive, bechtold20163d, nuchitprasitchai2017factors}. The expiration of patents in combination with low-cost 3D printer initiatives influenced the recent re-emerge of interest \cite{beaman2014additive} around 3D printing and caused the upcoming of personal 3D printers \cite{bechtold20163d}. The focus of this paper are potentially disruptive \cite{sauramo2014proliferation} personal 3D printers \cite{bechtold20163d}. Personal 3D printers refer to the low-cost 3D printers which emerged out of the research initiatives, predominantly bought by end-users in contrast to industrial scale 3D printers \cite{sauramo2014proliferation}. Fused Deposition Modeling (FDM), is a type of 3D printing technology that prints objects layer-by-layer. Most personal 3D printers are FDM-based. Therefore, the focus of this paper are FDM-based personal 3D printers. While personal 3D printers are now capable of printing objects with satisfactory quality \cite{gordeev2018improvement}, the print quality highly depends on the chosen print settings \cite{gordeev2018improvement, khan2020real, hernandez2015factors, jin2020automated}. These print settings are configured and optimized manually. There exists no widely adopted automatic optimization or generation of optimal, or at least better, print settings. The current process can be illustrated as follows: 

\begin{enumerate}
    \item 
    Print an object.
    
    \item
    Inspect the printed objected and determine potential causes for print defects. 
    
    \item
    Repeat steps 1 and 2 until the print quality is satisfying (enough).
\end{enumerate}

The illustrated process requires tedious manual labor and can take multiple hours but is the current state of the art \cite{khan2020real, hernandez2015factors}. Not only is that process a barrier in terms of ease-of-use for wider adoption of personal 3D printers \cite{sauramo2014proliferation}, but, and more importantly, the resulting unoptimized print settings lead to low(er) print quality. Low print quality is one of the predominant reasons against wide mass adoption of personal 3D printers \cite{sauramo2014proliferation, khan2020real, bechtold20163d, gordeev2018improvement, nuchitprasitchai2017factors}. The goal of this paper is the (semi-)automatic optimization of print settings for personal 3D printers in order to improve the print quality and by that potentially reduce the barrier for wider adoption.