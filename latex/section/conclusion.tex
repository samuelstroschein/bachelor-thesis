\section{Conclusion}

The developed optimization algorithm comes close to the optimal solution in a simulated environment. However, under real circumstances, the optimization process takes too long for too little visible improvement. One optimization process (≈15 test prints) easily takes 6 hours. Using standard, or community optimized, 3D print settings and optimize one specific print setting that is evaluated by detected print defects (defect gets less/more severe) or special tests like \cite{teachingTechCalibration} instead of optimizing multiple print settings simultaneously yields visible results faster. Therefore, the real-world application of the developed optimization algorithm, in its current form, is determined to be little to non-existent. 

Further research should subsequently focus on reducing the real-world optimization time. One solution could be a "lock" mechanism. If a print temperature of 205\textdegree C has been identified as optimal, no further print settings should be probed with a temperature other than 205\textdegree C; slowly reducing the amount of in-optimization print settings. Another solution could be to probe settings in batches and print test towers objects, as seen in figure \ref{figure/stringing}, instead of printing probed print settings "one by one". The amount of manual labour and time between each print is reduced by a lower amount of total prints. However, probing settings in batches could have a negative impact on the optimization process itself. Utilizing computer vision seems more promising in the context of too-long real-world optimization processes. Print defects could be detected in real-time, as some researchers already have done \cite{khan2020real}. Detecting and acting upon print defects in real-time could drastically lower the duration of an optimization process. 

Regardless, the limiting factor of real-world application might not be the optimization time alone. The optimization of a small subset of print settings did not lead to an order of improvement as anticipated and might have been a "dead end" from the start. A computer vision approach would have likely stumbled upon the same limitation. Although the detection of print defects in real-time through computer vision could be of enough value since a user, or the machine could interrupt an ongoing 3D print, reducing the amount of filament and time wasted on a defectious object. The tool Spaghetti Detective does just that with a very specific print defect and has tremendous success \cite{spaghettiDetective}. Furthermore, the print quality of personal FDM 3D printers might not be substantially improvable with software alone. Focusing upon improving the hardware of personal 3D printers, likely in combination with software, seems more promising. Development and adoption of entirely new print techniques are already on their way. While FDM-based personal 3D printers have been the subject of this paper, Resin-based 3D printers have become popular and cheap within the last two years. Resin 3D printers do not print an object layer by layer. They utilize light in the form of a laser or LCD screen. The advantages are fewer moving parts and unmatched print quality. The downsides are (even) longer printing times, the use of toxic chemicals and the requirement to clean the 3D printer after each print. However, if Resin-based, or other technologies, overcome their current downsides, FDM-based personal 3D printers might be replaced as "the standard personal 3D printer technology" and with its replacement, the problem of low quality 3D prints, influenced by none optimal print settings, might become a relict of the past.
